\chapter{Try Harmony in 2 minutes}

This brief chapter describes the few steps required to try Harmony on your computer. Harmony is written in Java. You must have at least Java 1.7 to run Harmony. You can download it from: \url{http://www.java.com/fr/}. Once it is done or if you already have Java on your system follow these simple steps:
\begin{itemize}
	\item Download the Harmony executable from this repository: \\\url{https://code.google.com/p/harmony/downloads/list}
	\item Extract the archive
	\item Run the native launcher of the application located in the extracted folder:
		\begin{itemize}
			\item \emph{Harmony.exe} on Microsoft Windows
			\item \emph{Harmony} on Linux
		\end{itemize}
	You should now have loaded all the Harmony components and see the \emph{OSGi} prompt : "osgi>". If you have an error check that your path is correctly set (see \url{http://www.java.com/fr/download/help/path.xml} for help).	
	\item Finally just type \emph{harmony} and wait for the default analysis to be executed.
\end{itemize}

Congratulations, you have just run your first Harmony analysis. You should have now an output directory named \emph{out} that contains the results of this analysis. The default analysis is called \emph{reporting} and generates several graphs on relationships between authors, events or item (see Harmony model for more explanation). The default source was used here, if you want to test Harmony on your own source repository the next chapter is for you.
	



