

\usepackage{listings}
\usepackage{xcolor}

\colorlet{punct}{red!60!black}
\definecolor{background}{HTML}{EEEEEE}
\definecolor{delim}{RGB}{20,105,176}
\colorlet{numb}{magenta!60!black}
\definecolor{grisbg}{gray}{0.95}

\lstdefinelanguage{json}{
    %basicstyle=\normalfont\ttfamily,
    numbers=none,
    %numberstyle=\scriptsize,
    stepnumber=1,
    %numbersep=8pt,
    showstringspaces=false,
    breaklines=true,
    frame=lines,
    backgroundcolor=\color{background},
    literate=
     *{0}{{{\color{numb}0}}}{1}
      {1}{{{\color{numb}1}}}{1}
      {2}{{{\color{numb}2}}}{1}
      {3}{{{\color{numb}3}}}{1}
      {4}{{{\color{numb}4}}}{1}
      {5}{{{\color{numb}5}}}{1}
      {6}{{{\color{numb}6}}}{1}
      {7}{{{\color{numb}7}}}{1}
      {8}{{{\color{numb}8}}}{1}
      {9}{{{\color{numb}9}}}{1}
      {:}{{{\color{punct}{:}}}}{1}
      {,}{{{\color{punct}{,}}}}{1}
      {\{}{{{\color{delim}{\{}}}}{1}
      {\}}{{{\color{delim}{\}}}}}{1}
      {[}{{{\color{delim}{[}}}}{1}
      {]}{{{\color{delim}{]}}}}{1},
}


% Import a source file to create a listing : [source file path] [title][label][langage name]
\newcommand{\includeSourceFile}[4]
{
	% On définit d'abord les propiétés du listing
	\lstset{basicstyle=\footnotesize\ttfamily,
			language=#4,
			float=!h,
			%%frame=tb,
			breaklines=true,
			frame=single,
			frameround = tttt,
			showspaces=false,
			showtabs=false,
			showstringspaces=false,
			backgroundcolor=\color{grisbg},
			numbers=none,
			tabsize=2,
			%stepnumber=5,
			%numberfirstline=true,
			%numberstyle=\tiny,
			keywordstyle=\color[rgb]{0,0,1},
        	commentstyle=\color[rgb]{0.133,0.545,0.133},
        	stringstyle=\color[rgb]{0.627,0.126,0.941}
      }
    % Puis on importe le fichier source
	\lstinputlisting[caption=#2,captionpos=b, label=#3]{#1}
}